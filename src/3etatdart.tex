\section{Etat de l 'art}
   Cette partie concerne plus l architecture global des clouds. Leurs évolutions 
   au cours du temps ce qui mettra le point sur le besoin de la création d un nouveau 
   point d entrée.   

    \subsection{Le premier Cloud de MDI}
        \gls{CC} est le premier cloud fait maison de \gls{mdi} qui englobe un ensemble 
        de services pour les clients... L'architecture globale de ce cloud est comme le montre 
        le shema suivant: 

        \begin{figure}[ht]
            \centering
            \includegraphics[scale=0.8]{\images/mos_overview.png}
            \caption{Aper\c cu MOS}
        \end{figure}

        \gls{CC} comporte plusieurs composants : 
        \begin{itemize}
            \renewcommand{\labelitemi}{$\bullet$}
            \item  un \gls{BS} qui est le composant d'échange de messages entre le cloud et 
            les OBD. Il consiste le point d'entrée du cloud.
            \item  un broker de message pour les systèmes Big Data
            \item  un ensemble de services qui sont accéder à partir de APIV3 
            \item une base de données pour le stockage des données 
            \item un dashboard pour l affichage des données et l acces visuel 
            des services au clients 
        \end{itemize}
       
    \break

    \subsection{De \gls{CC} à \gls{CN}}
        Exprimer le besoin du changement d'architecture. 


       

        \vspace{0.2cm}

       

    \subsection{Communication entre composants}
       mos  présente deux méthodes principales de communication entre les
        composants:
        \begin{itemize}
            \renewcommand{\labelitemi}{$\bullet$}
            \item Communication par évènements: Un composant produit un Event,
                et tous les listeners le re\c coivent. Cela est fait par l'intermédiaire
                d'un broker.
            \item Appels API: Chaque composant expose une API pour les composants
                avec authorisation. Le composant a plusieurs choix de méthodes de
                communication (MsgPck, TCP Socket).
        \end{itemize}

        \vspace{0.2cm}

        \begin{figure}[ht]
            \centering
            \includegraphics[scale=0.8]{\images/broker_pattern.png}
            \caption{Le broker MOS}
        \end{figure}

        \subsection{format de données du cloud }

        Track : 
        Message : 
        

        message track 

        Paragraphe sur le language Go : 
        Go\cite{rust_site} est un jeune language de programmation système
        fascinant. Ce language compilé hérite des idées des paradigmes de
        programmation impérative et fonctionnelle. Il définit des concepts et
        des règles qui assurent à l'utilisateur des binaires "safe".\\[0.3cm]
        Rust ne sacrifie pas la performance contre la sûreté. Il a des
        performances similaires à celles de C et C++. En fait, ses plus grands
        partisans espèrent qu'il va détrôner C!
      
