\section{itération 1 : nouveau BS au standard MQTT}
    \subsection{Conception de la première itération}
        \gls{mdi}, étant une entreprise innovante en télématique, doit toujours
        suivre et avancer les nouvelles technologies. Une de ces technologies
        est le protocol de communication standard des objets connectés: le MQTT.

        ...
        


    \subsection{Descriptif du protocole:Le standard MQTT, l'essentiel à savoir}
        Voir les couches de MQTT
        Ce protocole s'intéresse à toutes les couches de communication du modèle OSI.
       
        \subsubsection{La couche physique}
           v2x définit un protocole dérivé du wifi: le 802.11p, nommé 
            Ce protocole plus lent, mais plus adapté sert aux communications entre
            véhicules à haute vitesse, et entre ces véhicules et des stations intelligentes.
            Les connexions 3G et 4G ont l'inconvenient d'une grande latence, et sont donc
            éliminées comme vecteurs pour le protocole. En effet, a besoin
            d'établir des connexions directes entre véhicules, pour assurer une latence
            petite, et bornée.
            Une communication par 5G, la C-V2X (Cellular V2X), est envisagée vers
            2025: les specs du 5G répondent aux demandes de latence pour 
            pour cette couche.


        \subsubsection{La couche transport}
             utilise les protocols TCP et UDP sur IPv6. Il utilise en plus
            le protocol  Dans un premier temps, \gls{mdi}
            s'intéresse à un modèle sansActuellement, la couche transport
            est implémentée par la stack Marben.

       

    \subsection{Specifications techniques}
        La spécification téchnique peut être divisée en deux parties:
        \begin{itemize}
            \renewcommand{\labelitemi}{$\bullet$}
            \item La spec technique du protocole
            \item La spec ajoutée par \gls{mdi} pour adapter
        \end{itemize}
        \bigskip
       

    \subsection{Implémentation}
    Imlémenter c est faciiile !! 
        \subsubsection{Recherche \& état de l'art}
            

           
        \pagebreak

        \subsubsection{Développement effectué}
            Le développement de ce projet est divisé 

    \subsection{Tests \& Performance}
       
