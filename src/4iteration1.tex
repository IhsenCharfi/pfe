\section{itération 1 : nouveau BS au standard MQTT}
        \gls{mdi}, étant une entreprise innovante en télématique, doit toujours
        suivre et avancer les nouvelles technologies. Une de ces technologies
        est le protocol de communication standard des objets connectés: le MQTT.

        ...
        


    \subsection{Descriptif du protocole:Le standard MQTT, l'essentiel à savoir}
    D’apres la specification officielle MQTT \cite{mqtt_site},
    le standard MQTT est un protocole de transport de message Client/serveur sous le pattern de Publication/Souscription 
    “ Publish/ Subscribe”. 
    C'est un protocole léger , ouvert, simple et désigné d’être facile à implémenter. \\
    Ces caractéristiques le rend idéal à être utilisé dans plusieurs situations, surtout pour la communication Machine 
    to MAchine M2M et pour l’internet des objets IoT. Des contextes où la faible empreinte du code est requise ainsi 
    qu’une bonne bande passante.     \\
    
       
    \textbf{Qu’offre MQTT ? }\\

    \begin{itemize}
        \renewcommand{\labelitemi}{$\bullet$}
        \item L’utilisation du pattern de message “ Publication/Souscription” qui offre 
        des distributions un-à-plusieurs ( one-to-many) et le découplage de ces deux parties.    
        \item 3 Qualités de service QoS pour la distribution des messages . 
        \item Un échange de protocol minimisé pour réduire le trafic  du réseau      
        \item Un mécanisme pour notifier les parties intéressées lors d'une déconnexion anormal    
    \end{itemize}   
    \bigskip 
    En cas de besoin de plus de détails sur le standard, consultez la partie Annexes. Elle comporte des eclaicissements 
    sur le standard MQTT.

       

    \subsection{Specifications techniques \& Implémentation}
       
        Le Point d’entrée du cloud ou BS se charge de deux principales fonctionnalités. 
        \begin{itemize}
            \renewcommand{\labelitemi}{$\bullet$}
            \item  \textbf{Transfert de données :} où ils gèrent les connexions avec les boîtiers.
            \item  \textbf{Traitement de données :} encodage/décodage des données de messages selon le format de données du cloud.
        \end{itemize}

        Puisque le standard repose sur le découplage entre le publisher et subscriber, la conception doit prendre 
        en compte  un broker MQTT pour assurer cette notion. 
        La conception est présenté par la figure suivante .. 

 ********* Insertion d une figure de la conception ************** \\
        Le BS aura deux parties importantes qui sont le transfert des données et l'encodage pour le broker.

        \subsubsection{Recherche \& état de l'art}
            
        \textbf{A la recherche d'un broker MQTT! }

        Il existe quelque large implémentation de MQTT comme Facebook Messenger par example, mais aussi de nombreux 
        outils monétisés et d'autres open source.
        Il y a aussi un projet Eclipse active, "Paho", qui offre une implémentation scalable , open source pour 
        différents langage de programmation comme Java , 
        C, C++ , Python , JavaScript , C\# et Go lang. \\
        Il existe plusieurs implémentations qui sont réunis dans cette référence et qui les comparent selon plusieurs critères. 
        Pour effectuer un bon choix satisfaisant de toute ces propositions, il faut bien demander les critères de l'entreprise en premier lieu 
        sur lesquels on se base. \\ \break 
        Comme la stratégie de l entreprise est de réduire les coûts, nous éliminons les choix d'outils payants. 
        Un autre critère est la charge des connexions du broker. L'un des objectifs de \gls{md30} est d'augmenter le support 
        en charge du nouveau BS ce qui revient dans ce cas à étudier la charge du broker MQTT en question. 
        Ceci peut être effectué de deux manières : 
        \begin{itemize}
            \item Augmenter le support en charge du Broker par rapport au \gls{BS} de \gls{mdi21}, sachant que
            le support du BS actuel est de ... 
            \item Assurer la scalabilité du broker ce qui revient à choisir dés le début un broker scalable.
        \end{itemize}
        
        
        

         
           
        \pagebreak

        \subsubsection{Développement effectué}
            Le développement de ce projet est divisé 

    \subsection{Tests \& Performance}
       


    Paragraphe sur le language Go : 
    Go est un jeune language de programmation système
    fascinant. Ce language compilé hérite des idées des paradigmes de
    programmation impérative et fonctionnelle. Il définit des concepts et
    des règles qui assurent à l'utilisateur des binaires .\\[0.3cm]
   
  
    Le but de cette explication est de montrer la différence considérable
    avec d'autres langages de programmation. Un lecteur intéressé pourra
    apprendre le langage pour plus d'informations.

